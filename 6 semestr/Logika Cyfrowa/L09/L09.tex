\documentclass{article}
\usepackage{listings}
\usepackage[T1]{fontenc}
\usepackage{titlesec}
\usepackage{graphicx}
\usepackage{amsmath}
\usepackage{xcolor}
\usepackage{amssymb}
\usepackage{circuitikz}
\usepackage{trace}
%\usepackage{minted}
\titleformat{\section}  % which section command to format
  {\fontsize{10}{12}\bfseries} % format for whole line
  {\thesection} % how to show number
  {1em} % space between number and text
  {} % formatting for just the text
  [] % formatting for after the text
  \title{Logika Cyfrowa}
\author{Jakub Gałaszewski} 
\begin{document}
\maketitle
\section{Narysuj schemat obwodu implementującego deterministyczny automat skończony (lub automat Moore’a z jednobitowym wyjściem) opisany poniższą tabelą stanów, ze stanem akceptującym 11:}
\begin{center}
	\begin{tabular}{|c|c||c|} 
	 \hline
	$q$ & $a$ & $q_0$ \\ 
	 \hline \hline
	 00&0& 10\\ \hline
	 01&0& 01\\ \hline
	 10&0& 11\\ \hline
	 11&0& 10\\ \hline
	 00&1& 11\\ \hline
	 01&1& 00\\ \hline
	 10&1& 00\\ \hline
	 11&1& 01\\ \hline
	\end{tabular}
\end{center}

\begin{center}
	$q_{O0}$\\
	\begin{tabular}{|c|c|c|c|c|} 
	 \hline
	$a\backslash q_1q_0$ &$00$ & $01$ & $11$ & $10$ \\ 
	 \hline \hline
	 0&0&1&0&1\\ \hline
	 1&1&0&1&0\\ \hline
	\end{tabular}\\
	$q_{O0}=q_1 \oplus q_0 \oplus a$\\
\end{center}
\begin{center}
	$q_{O1}$\\
	\begin{tabular}{|c|c|c|c|c|} 
	 \hline
	$a\backslash q_1q_0$ &$00$ & $01$ & $11$ & $10$ \\ 
	 \hline \hline
	 0&0&1&0&1\\ \hline
	 1&1&0&1&0\\ \hline
	\end{tabular}\\
	$q_{O1}=\bar{q_1}\bar{q_0} + \bar{a}q_1$\\
\end{center}
TODO RYSUNEK
\section{Zdefiniuj deterministyczny automat skończony (np. używając tabeli stanów lub diagramu stanów) rozpoznający ciągi bitów kończące się sekwencją 1001 lub 1111 (czyli język (0 + 1)*(1001 + 1111)). Przykładowo, automat powinien rozpoznać ciąg 010111100110011111 oraz jego prefiksy: 01011110011001111, 01011110011001, 0101111001, 0101111.}
\begin{center}
	\includegraphics[scale=0.6]{"L09Z02.png"}
\end{center}
\section{Zaprojektuj obwód, którego wyjście o jest w stanie wysokim wtedy i tylko wtedy, gdy przez cztery kolejne cykle zegara dwa bity wejściowe w1 i w2 są równe.}
\begin{center}
	\includegraphics[scale=0.09]{"L09Z03.jpg"}
	\includegraphics[scale=0.15]{"L09Z03II.jpg"}
\end{center}
\section{Zdefiniuj automat Mealy’ego rozpoznający ciągi bitów, które wśród trzech ostatnich bitów mają nieparzystą liczbę jedynek (czyli język 1 + 01 + 10 + (0 + 1)*(001 + 010 + 100 + 111)).}
$Q =  \mathbb{B}^3$,
$\Sigma =  \{0,1\}$,
$\Omega =  \{0,1\}$,
$\delta =  f(x_2,x_1,x_0, a) = x_1x_0a$,
$\chi =  f(x_2,x_1,x_0,a) = \oplus x_1 \oplus x_0 \oplus a$
\section{Zdefiniuj automat Moore’a rozpoznający, czy liczba jedynek na wejściu jest podzielna przez 4 lub o 1 większa od liczby podzielnej przez 4.}
$Q =  \{0, 1, 2, 3 \}$,
$\Sigma =  \{0,1\}$,
$\Omega =  \{0,1\}$,
$\delta =  f(x, a) = begin(cases)
x jeśli a = 0\\
(x+1)mod 4 w p.p.$,
$\chi =  f(x) =  begin(cases)
1 jeśli x mod 4 = 1 lub x mod 4 = 0\\
0 w p.p.$
\end{document}
